-------------------------------------------------------------------------------------

Стр. 6

Глава 1: Введение в Git

Итак, вы хотите изучить Git? Тогда вы обратились по адресу. В этой электронной книге
я проведу нить разговора от незначительных вещей до управления ваших проектов с
помощью Git. Но перед тем как мы начнём, давайте сделаем шаг назад и уясним что
именно такое Git и почему вы захотите его использовать.

-------------------------------------------------------------------------------------

Стр. 7

[ Что такое Git? ]

Git -- это менеджер исходных текстов. Правильнее, это распределённая система
контроля версий. Но что именно это значит? Давайте разберёмся.

[ Git -- это ПО для разработки ПО ]

Во-первых, Git -- это не язык, концепция или рекомендация. Это программа, один из
инструментов, который вы можете использовать в вашей разработке так же, как и 
текстовый редактор или FTP-клиент. Итак, для чего это всё? Git управляет вашим 
исходным кодом...но что это значит?

[ Git предназначен для версирования ]

Идея, стоящая за Git (и другими менеджерами исходных текстов, о которых мы поговорим
позже) -- это разумная идея сохранения "снимков" ваших кодируемых проектов. Это 
значит, что вы определяете моменты в истории развития проекта. Составляете шкалу,
где все метки представляют этап в вашем кодировании. Где вы расставите эти меткиO
зависит от вас, но разумно ставить их там, где вы что-то завершили. Например, вы
могли бы делать метки после каждой реализованной функции. Таким образом, шкала
вашего проекта могла быть похожа на что-нибудь такое:
    * метка 1 - старт проекта, добавлены CSS и JS файлы
    * метка 2 - построена основная структура веб-сайта
    * метка 3 - добавлена навигация
Надеюсь вы прониклись идеей "снимков" исходного кода. Заимствуя пример Тома
Престона-Вернера (ключевая фигура Git сообщества), это как фотографирование вашего
ребёнка каждый код 

-------------------------------------------------------------------------------------

Стр. 8

для отслеживания его взросления. Итак, если вы обратите внимание
на концепцию меток, я думаю вы увидите для чего подходит Git: если вы захотите делать
эти метки самостоятельно, то вам понадобится копировать директорию с проектом и
переименовывать каждый раз. Git делает это, и даже больше, за вас.

[ Git -- это распределённая система ]

Именно по этому Git называется системой контроля версий. Но она также и
распределённая. Распределённость в системах контроля версий значит что репозиторий
исходного кода (то, что является вашим проектом с папками) не нуждается в конкретном
месте хранения. В некоторых других системах главный репозиторий хранится на сервере
и вы как программист просто получаете наиболее свежую версию проекта для дальнейшей
работы. Затем когда вы хотите создать новый "снимок", вы отправляете его обратно на
сервер.
Давайте остановимся на секунду: системы контроля версий создаются так, чтобы
несколько человек могли работать над одним проектом одновременно. Каждый программист
получает код, работает над ним и делает "снимки" (которые правильно называются 
коммиты). Затем они могут поделиться своими коммитами с другими.
Итак, вы можете сделать коммит и отослать обратно на сервер. Это НЕ распределённая
система. Она требует доступа к серверу для создания коммита или его получения от
остальных (тех, кто отправил свои изменения на сервер). В Git просто всё по-другому.
Почти всё, что вы делаете с Git происходит на вашей собственной машине. Конечно, вы
всё ещё можете делиться вашим репозиторием с другими, но распределённым способом:
вы можете отсылать коммиты напрямую другим компьютерам и получать коммиты обратно от
них. Также каждая копия репозитория содержит полную историю коммитов с Git. Это не
так для нераспределённых (или централизированных) систем, где вы получаете только
последний коммит от сервера. Распределённый способ безопаснее в том, что каждая
копия репозитория является полной копией и не будет большой утратой если одна из
них испортится. В централизированных системах если сервер выходит из строя, то

-------------------------------------------------------------------------------------

Стр. 9

история коммитов тоже пропадает.

[ Почему мне следует использовать систему контроля версий? ]

Мы уже увидели, что системы наподобии Git упрощают создание шкалы изменеий кода, но
почему мы захотим это делать? И есть ли другие преимущества? Давайте обсудим
несколько причин почему вам следует использовать менеджеры исходного кода.

[ Свобода действий ]

Когда вы используете систему контроля версий и создаёте коммиты регулярно, то вам нет
смысла беспокоиться если что-либо сломается. Если у вас действительно что-то
перестало работать, просто откатитесь назад к последнему коммиту и продолжайте.

[ Свобода ветвлений ]

Более подробно о ветвлении мы поговорим в третьей главе, но кратко это позволяет вам
напрявить ваш проект в двух или более плоскостях в пределах одного репозитория
(поверьте мне, это покажется волшебством). Это может понадобиться в случаях когда
вы строите вторую версию проекта, но в то же время нужно исправлять ошибки в первой 
версии. Или может быть вы хотите реализовать достаточно спорную функцию. Это было бы
определённо неплохим ходом выделить для этого собственную ветку. Без ветвлений вам 
приходилось бы дублировать директоию с проектом всяктй раз когда вам захотелось
попробовать чего-нибудь новое: грустно.

[ Свобода публикаций ]

Использование систем контроля версий действительно упрощает публикацию вашего проекта
для других и позволяет им помогать вам в его разработке. Без таких систем вам
приходилось бы 

-------------------------------------------------------------------------------------

Стр. 10

копировать, сравнивать и интегрировать все их изменения вручную. Опять
же грусто.

[ Откуда пришёл Git? ]

Время для урока истории: Git создан Линусом Торвальдсом, автором ядра Linux. На самом
деле Линус создал Git для управления исходным кодом ядра Linux. Он обозрел
существующий менеджеры исходного кода и пришёл к выводу, что ни один из них
достаточно быстрый. Поэтому от сделал свой собственный. Вы можете обрести душевное
спокойствие в том, что Git в состоянии обрабатывать ваши проекты с молниеносной
скоростью, если вы не строите ОС.

[ Почему бы не использовать другой менеджер исходного кода? ]

Надеюсь сейчас вы убедились, что это хорошая идея использовать систему контроля
версий. Но почему Git? Есть несколько других вариантов, основные из которых:
    * Subvesion
    * Mercurial
    * Perfoce
    * Bazaar
Что с ними не так? На самом деле с ними всё в порядке, но есть несколько причин по
которым вы можете предпочесть Git:
    * Git быстрее
    * Git проще в изучении
    * Git предлагает временную область

-------------------------------------------------------------------------------------

Стр. 11

    * GitHub доступен для публикаций
Есть и другие причины, но они не такие значительные, если вы только осваиваетесь в
Git. Вы можете прочитать о них и о других деталях на WhyGitIsBetterThanX.com. Как
только у вас станет получаться с Git, вернитесь назад и изучите сайт. Когда я так
сделал, я был удивлён, увидев, что другие системы управления исходным кодом так 
уступают.

[ Итог ]

В этой главе мы узнали что такое Git, откуда он пришёл и почему вам следует его 
использовать. Сейчас давайте погрузимся в терминал и разомнём свои пальцы 
несколькими командами.