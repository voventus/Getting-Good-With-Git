-------------------------------------------------------------------------------------

Стр. 6

Глава 1: Введение в Git

Итак, вы хотите изучить Git? Тогда вы обратились по адресу. В этой электронной книге
я проведу нить разговора от незначительных вещей до управления ваших проектов с
помощью Git. Но перед тем как мы начнём, давайте сделаем шаг назад и уясним что
именно такое Git и почему вы захотите его использовать.

-------------------------------------------------------------------------------------

Стр. 7

Что такое Git?
Git -- это менеджер исходных текстов. Правильнее, это распределённая система
контроля версий. Но что именно это значит? Давайте разберёмся.

Git -- это ПО для разработки ПО
Во-первых, Git -- это не язык, концепция или рекомендация. Это программа, один из
инструментов, который вы можете использовать в вашей разработке так же, как и 
текстовый редактор или FTP-клиент. Итак, для чего это всё? Git управляет вашим 
исходным кодом...но что это значит?

Git предназначен для версирования
Идея, стоящая за Git (и другими менеджерами исходных текстов, о которых мы поговорим
позже) -- это разумная идея сохранения "снимков" ваших кодируемых проектов. Это 
значит, что вы определяете моменты в истории развития проекта. Составляете шкалу,
где все метки представляют этап в вашем кодировании. Где вы расставите эти метки
зависит от вас, но разумно ставить их там, где вы что-то завершили. Например, вы
могли бы делать метки после каждой реализованной функции. Таким образом, шкала
вашего проекта могла быть похожа на что-нибудь такое:
	* метка 1 - старт проекта, добавлены CSS и JS файлы
	* метка 2 - построена основная структура веб-сайта
	* метка 3 - добавлена навигация
Надеюсь вы прониклись идеей "снимков" исходного кода. Заимствуя пример Тома
Престона-Вернера (ключевая фигура Git сообщества), это как фотографирование вашего
ребёнка каждый код 

-------------------------------------------------------------------------------------

Стр. 8

для отслеживания его взросления. Итак, если вы обратите внимание
на концепцию меток, я думаю вы увидите для чего подходит Git: если вы захотите делать
эти метки самостоятельно, то вам понадобится копировать директорию с проектом и
переименовывать каждый раз. Git делает это, и даже больше, за вас.