-------------------------------------------------------------------------------------

Стр. 27

Глава 3: Конфигурация

Что ж, сейчас вы знаете как мы будем использовать Git и вам комфортно в командной
строке. Сейчас мы фактически начнём использовать Git. Конечно, мы начнём с его
установки!

-------------------------------------------------------------------------------------

Стр. 28

[ Установка Git ]

Установка Git очень сильно зависит от используемой ОС, такчто я сначала покажу как
это сделать сначала в Mac, затем в Windows.

[ MacOS ]

Существует несколько способов установки Git в Mac: самый простой способ это 
использовать Git OS X Installer. Зайдите на git-osx-installer page в Google Code и 
кликните "Загрузить пакет". Затем загрузите свежий образ, в моём случае это Git 1.7.1.
Смонтируйте образ у установите содержащиеся внутри образа пакеты. Если вы планируете
использовать Git с помощью программ, отличных от терминала, запустите Shell скрипт
в образе.
Если вы поьзуетесь менеджнром пакетов, таким как Homebrew или MacPorts, то это
наипростейший способ установки Git.

-------------------------------------------------------------------------------------

Стр. 29

(Homebrew)

\$ sudo brew install git

(MacPorts)

\$ sudo port selfupdate
\$ sudo port install git-core

(Часть с sudo просто значит запуск команды от пользователя root. Вы обычно будете это
делаеть, когда устанавливаете ПО. Когда используете sudo, командная строка запросит
пароль вашего аккаунта.)
Если вы хотите действительно сложный способ, вы можете установить Git из исходных
кодов. НО сначала вам при дётся установить Apple Developer Tools. Вы можете получить
их ан http://developer.apple.com/mac/. Они находятся на правой боковой панели (Если
у вас не Snow Lopard, установите их с CD/DVD для вашей версии Mac OS X). Как только
всё сделаете, можете продолжать.
Мы собираемся устанавливать Git в директорию /user/local, таким образом убедитесь, 
что терминал сможет его найти в переменной \$PATH. Используя команды о которых 
мы говорили выше, проверьте в ней путь /usr/local/bin (bin означает двоичный и это 
поддиректория, в которой сохранён откомпилированный Git). Если пути нет, то
откройте файл .profile и добавьте такую строчку в конец
export PATH=”/usr/local/bin:\$PATH”

-------------------------------------------------------------------------------------

Стр. 30

        Подсказка от Rock*
        Что такое .profile? Терминал Mac испоьзует файл под названием .profile для 
        хранения списка настроек командной строки. Как только вы сильнее привыкните
        к командной строке, вы заметите, что обращаетесь к этому и другим файлам для
        редактирования. Файл /.profile находится в вашей домашней директории, но вы 
        не сожете увидеть его по умолчанию. Самый простой способ получить к нему 
        доступ через терминал с помощью команды mate ~/.profile. Замените слово mate 
        на выбранный вами редактор.
После того, как вы сохранили и закрыли файл .profile, перезапустите терминал.
Выполнение echo \$PATH покажет вам, что директория /usr/local/bin теперь в перменной
окружения \$PATH.
Сейчас мы готовы к установке Git. Мы выполним несколько команд, которые мы не обсуждали, 
так что если вас не устраивает такое положение, то вам, вероятно, следует использовать
один из предыдущих вариантов. Если вы решились, просто будьте осторожны и пусть эти 
команды служат отправной точкой в ваших исследованиях командной строки!
Мы начнём  с загрузки и распаковывания исходного кода Git:
\$ curl -O http://kernel.org/pub/software/scm/git/git-1.7.1.tar.bz2
\$ tar xzvf git-1.7.1
Сейчас мы перейдём в эту папку и соберём Git:
\$ cd git-1.7.1
\$ ./configure --prefix=/usr/local
\$ make
\$ make install
Готово! Сейчас вы можете удалить папку и архив:
\$ cd ..
\$ rm git-1.7.1
\$ rm git-1.7.1.tar.bz2

-------------------------------------------------------------------------------------

Стр.31

Теперь вы готовы к работе. Вы сможете убедиться, что Git успешно установился 
выполнением команды:
\$ which git
Вы получите /usr/local/bin/git. Если это так, то вы удачно установили Git из исходного 
кода! (Спасибо Дэну Бенжамину за эту инструкцию по устанвке Git из исходных кодов).

[ Windows ]

В связи с корнями от Linux, Git не работает как "родной" в Windows, как и на Mac. Тем 
не менее, пользователи ПК могут воспользоваться преимуществами Git. Вот как это 
делается.
Мы установим Git на Windows, используя установщик Msysygit, который вы сможете загрузить
на странице проекта в Google Code. Загрузите последнюю версию со страницы загрузки, 
которая, кажется, под номером 1.7.0.2 на момент написания статьи (помечена как бета-
-версия). Запустите установщик и следуйте указаниям. Это будет похоже на обычный процесс
установки приложения до тех пока вы не прийдёте к этому окну:
{рисунок}

-------------------------------------------------------------------------------------

Стр. 32

Оно спрашивает как вы хотите интегрировать Git в вашу систему. Если вы пробежитесь по 
опциям, то увидите, что по умолчанию Msysgit создаст отдельную командную строку из 
которой вы можете выполнять команды Git. Конечно, вы сможете выполнять все обычные
команды тоже. Вторя опция включит путь до утилит Git в переменную окружения PATH, 
позволяя вам выполнять команды Git из обычной cmd.exe. Последняя опция, дополненная
предупреждением, включит утилиты Git и Unix в переменную PATH. Это перезапишет утилиты
командной строки Windows, которые имеют те же имена, что и в Unix, так что убедитесь,
что вы этого хотите при выборе этой опции. Так как я предпочёл утилиты Unix, то я
выбрал третью опцию. Если вы не уверены, то вторая  опуця более подходящая, чем первая.
Следующее диалоговое окно может внести некоторую путаницу:
(рисунок)
Здесь говорится о невидимых символах, которые помечают окончания строк. Unix системы
требуют символ первода строки, в то время как Windows системы требуют символа возврата
каретки и первода строки.
Согласно диалоговому окну, опция "Извлекать в стиле Windows, фиксировать в стиле Unix"
лучший вариант для Windows, когда реализуются кросс-платформенные проекты. Если 

-------------------------------------------------------------------------------------

Стр. 33

вы работаете только на Windows машинах, можете выбрать опцию "Извлекать как есть, 
фиксировать как есть", но я не рекомендую этот способ. Вы никогда не знаете когда 
ваш проект понадобится отредактировать на другой ОС.
После этих решений установка Git завершена. Вы вероятно получите ярлых "Git Bash" на
вашем рабочем столе. Хотя, возможно, вы решили включить Git в переменную PATH, так что
вы можете использовать команды из обычной командной строки. Этот ярлык обеспечивает
несколько дополнительных функций, таких как подсветка текста и немного информации по Git
в командной строке. Как только вы приобретёте навык работы с Git, вы можете выбирать как
угодно.
Итак, сейчас вы установили Git в вашу систему. Время кое-что настроить до того как мы
начнём его использовать.

[ Настройка Git ]

Я знаю, мы движемся медленно, но есть ещё один шаг, который необходимо сделать, прежде
чем мы сможем начать. Мы настроем Git так, чтобы он знал кто вы.
Откройте вашу командную строку (в действительности вы не должны закрывать её до конца 
книги) и выполните это:
\$ git config --global user.name “Andrew Burgess”
\$ git config --global user.email “andrew8088@gmail.com”
Конечно, замените моё имя и e-mail на свои. Эта информация очень важна и вы увидите 
позже почему. Несмотря на то, что есть и другик парамеиры конфигурации которые вы можете
использовать, я сейчас установлю просто ещё один сейчас: все языки программирования 
которые я использую независимы от числа пробелов, поэтому я установлю:
\$ git config --global apply.whitespace nowarn

-------------------------------------------------------------------------------------

Стр. 34

Так Git будет игнорировать изменения пробелов. Существует много других конфигурационных
настроек, но они вам не нужны до более глубокого понимания Git. Если вы хотите о них
узнать, вы можете посмотреть их в официальной документации Git.