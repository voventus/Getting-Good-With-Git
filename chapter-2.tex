-------------------------------------------------------------------------------------

Стр. 13

Глава 2: Команды

По умолчанию Git является полностью консольным ПО, поэтому если вы хотите
использовать систему успешно, то вам необходимо легко ориентироваться в командной
строке. Если ваши навыки работы в ней не достаточно отточены, то эта глава поможет 
вам повысить свою скорость работы.

-------------------------------------------------------------------------------------

Стр. 14

[ Предупреждение ]

Прежде чем начнём что-либо выполнять в командной строке, вам необходимо кое-что
знать: 99\% всего что вы делаете необратимо. Это особенно касается операций удаления.
Когда вы удаляете файлы или папки из командной строки, то они не помещаются в корзину
или какое-либо другое место. Они исчезают навсегда (за исключением аппаратных 
способов восстановления...которые, однако, не всегда будут работать). Поэтому, 
разумно дважды проверять свои команды при использовании командной строки.
Работа в командной строке может быть более продуктивной и (если вы такой же гик, как 
и я) очень забавной. Не стоит пугаться непонятной сущности всего этого. Делайте всё
медленно и это сохранит ваше время.
Также сейчас не бойтесь, что что-либо сломается: все кроме одной команды, которые
мы обсудим, совершенно безвредны.
Есть ещё одна вещь на заметку: если команда была выполнена удачно, то она как правило
ничего не сообщает в ответ. Это может ввести в заблуждение, но примите как факт, что
если командная строка отвечает вам в ответ, то вам лучше обратить внимание: что-то
вероятно не так! (Конечно есть несколько команд, и в Git их тоже много, которые
специально генерируют выходную информацию).

[ Другое предупреждение ]

Git будет работать так же хорошо в одной ОС как и в другой. Тем не менее, остальные
команды командной строки отличаются от ОС к ОС. Всё, что мы будеи обсуждать здесь
будет хорошо работать в Linux дистрибутивах и в MacOS X. Если вы используете Windows
не беспокойтесь: большинство будет работать и здесь. Если будут какие-либо отличия я
буду конечно на них указывать.

-------------------------------------------------------------------------------------

Стр. 15

Ах да, если вы используете Linux, то пожалуйсто пропустите остаток этой главы: вы
абсолютно ничего нового не узнаете.

[ Запуск командной строки ]

Несмотря на то, что все поймут о чём вы говорите, когда произносите "командная 
строка", ни в OS X, ни в Windows это так не называется. В Mac это называется 
Terminal. Чтобы его запустить откройте /Applications/Utilities/Terminal.app (или 
просто сделайте Spotlight/Quicksilver поиск по "Terminal").
В Windows это называется Command Prompt. Для запуска в Windows XP выберите Пуск -> 
Программы -> Стандартные -> Command Prompt. Если вы на Windows Vista или Windows 7,
просто поищите "Command Prompt" или "cmd.exe" из стартового меню ("cmd.exe" из меню
"выполнить" запускается в любой версии).
Сейчас, когда вы смотрите на командную строку, давайте начнём.

-------------------------------------------------------------------------------------

Стр. 16

[ Что вы видите? ]

Итак, вы запустили свою командную строку. Не пугайтесь. Это вероятно похоже на это :
{рисунок}
Или на это:
{рисунок}

-------------------------------------------------------------------------------------

Стр. 17

Итак, что это значит? Давайте сначала посмотрим на Mac Terminal. Строка начинается с
имени компьютера: в нашем случае это andrew-inac. Затем после двоеточия идёт имя 
текущей директории (мы поговорим об этом дальше). После пробела идёт имя текущего
пользователя (andrew) и затем знак доллара. Заметьте, это приветствие может быть 
изменено, так что можете наблюдать приветствие по-разному.
В Windows Prompt вы просто получаете путь до текущей директории и знак "больше чем" (>).
После этой строки вы вводите одну или несколько команд. После нажатия "Enter" они 
выполнятся и вы получите другую подсказку, извещаюшую вас о том, что выполнение 
предыдущей команды завершено.

[ Команды ]

Давайте же перейдём к командам наконец!

(whoami)

Мы начнём с простой команды. Введите следующее в свою командную строку (замечание:
знак доллара в начале -- это моё представление вашей командной строки. Не печатайте
его):
{код}
Да, это "Who am I?". В OS X терминал напечатает ваше имя пользоваиеля. В Windows вы 
получите и имя пользователя, и имя компьютера.
Вам будет интересно почему мы получили ответ от команды. Запомните, существует два
типа команд: команды, которые сообщают вам что-либо (наподобии whoami) и команды,
которые делают что-либо (эти команды обычно ничего не сообщают в случае успеха).

-------------------------------------------------------------------------------------

Стр. 18

(pwd)

Дальше, если вы на Mac, то попробуйте эту команду:
{код}
Да, это что-то значит: вы попросили командную строку напечатать рабочую директорию
(Print Working Directory прим. переводчика) (Директория -- это альтернативное 
название для папки, к которому следует привыкнуть, если вы планируете много работать
в командной строке). Это направляет на другую особенность командной строки: когда вы
работаете в терминале, вы всегда работатете из какой-то папки. Команда pwd позволяет
определить в какой папке вы находитесь. В Mac вы можете получить такой ответ:
{код}
Выше я говорил, что имя вашей текущей папки (или директории) видно в подсказке. Ваша
домашняя папка представляется тильдой (~) и это всё, что показывается в вашей 
командной строке. Команда pwd возвращает полный путь.
Команды pwd нет в Windows Command Prompt...но вам это и не понадобится: полный путь
до текущей папки отображается в подсказке, поэтому вы можете всегда его видеть.

(ls / dir)

Итак, если вы всегда находитесь в папке, когда работаете в командной строке, то вы
вероятно работаете с содержимым этой папки. Поэтому вы захотите узнать её содержимое.
Для этого наша следующа команда: на Mac введите
{код}
Если вы на Windows сделайте так:
{код}

-------------------------------------------------------------------------------------

Стр. 19

Команда ls означает "list files", dir значит "directory". Обе эти команды по существу
делают одно и то же: показывают вам файлы в текущей директории. В Mac это вероятно 
похоже на такое:
{код}
А на Windows вы увидите это:
{код}
Информации, показываемой по умолчанию в Windows намного больше, но с некоторым
количеством опций в которые мы не будем сейчас вдаваться.
Раз уж мы здесь, давайте посмотрим на первые две записи в директории Windows. Это .
(точка) и .. (двойная точка). И в Windows, и в Mac одинарная точка представляет папку
в которой вы сейчас, а двойная точка представляет верхнюю папку или "родительскую"
папку той папки в которой вы сейчас. Это пригодится.

(cd)

Это простая команда: cd предназначен для смены директории (Change Directory). Мы
можем использовать эту команду для переключения между папками. Давайте предположим,
что директория в которой вы сейчас находитесь содержится директория "images". Чтобы
перейти в эту директорию просто наберите:
{код}

-------------------------------------------------------------------------------------

Стр. 20

Это первая рассмотренная команда, в которой мы использовали параметр, но параметры 
очень часто присутствуют в командной строке. Вы также можете использовать эту команду
для прыжка на более чем один уровень с помощью такого:
{код}
Но как мы вернёмся назад по файловой структуре? Помните ту двойную точку? Она
поднимет на папку:
{код}
Конечно вы можете подняться на несколько папок сразу таким же способом, каким
опустились: ../..
Ещё одну вещь отметим здесь: что произойдёт если мы не передадим команде cd имя
директории? В Mac это перекинет вас в домашнюю директорию (как и cd ~). В Windows
это активирует действие команды pwd как на Mac: печать пути до папки в которой вы 
сейчас.

(mkdir)

Итак, сейчас мы знаем как перемещаться между папкамии видеть что в них. А как мы 
можем создавать папки? И в Windows, и в Mac это очень просто:
{код}
Действительно это так просто.

(rm / del)

Одна команда которую мы будем смотреть, представляющая потенциальную опасность,
поэтому будьте осторожны, когда вы её используете. Команда rm предназначена для
удаления файлов и папок.

-------------------------------------------------------------------------------------

Стр. 21

{код}
Если вы попытаетесь удалить несуществующую папку, то вы получите сообщение об ошибке.
Это простой способ удалить папку с содержимым, но я оставлю этот опыт для вас.

В Windows вам придётся использовать команду del (считайте delete). Так же, как и в 
Mac, она работает и с файлами и папками:
{код}

(cp and mv / copy and move)

Следующий шаг про пермещение файлов. Опять же это действительно не так сложно. Вот
как копировать файлы: первый параметр -- это файл, который вы хотите скопировать, 
второй параметр -- целевое имя файла.
{код}
Если вы хотите переместить файл, то вы можете использовать тот же синтаксис и просто
заменить cp на mv.
{код}
Обратите внимание, что я не указал имя файла в конце пути. Если вы не хотите изменять
имя файла при перемещении, то вам не нужно добавлять имя файла в директории 
назначения. Это работает также и с командой cp.
        Подсказка от Rock*
        Команда mv может также использоваться и для переименования файлов: просто
        используйте mv oldFile.name newFile.name
В Windows эти команды являются полными названиями copy и move. Они работают почти так
же, как и в Mac, хотя вы 

-------------------------------------------------------------------------------------

Стр. 22

должны заметить, что путь в Windows требует обратный слэш
(\\). В Mac всё с прямым слэшэм (/).

[ Открытие файлов ]

Давайте сейчас поговорим об открытии файлов для редактирования. Когда вы в командной
строке есть два способа отредактировать файл:
    1. С помощью внешнего редактора
    2. С помощью окна терминала
Стандартный способ для всех редактроров открыть файл, использовать команду редактора,
за которой следует имя открываемого файла. Например, в Mac редактор TextMate 
определяет команду для открытия файлов из командной строки. Введите:
\$ mate index.html
Эта команда откроет файл index.html в TextMate. Существует множетсво редакторов и в
Windows, и в Mac, поэтому я не смогу затронуть их все. Обратитесь к Google и вы 
вероятно найдёте инстукцию к выбранному редактору.
        Подсказка от Rock*
        Если вы на Windows, то вы должны просто использовать путь до исполняемого 
        файла редактора с передачей редактируемого файла. Например: 
        \$ "C:\Program Files\Notepad++\notepad++.exe" default.css
Чтобы редактировать файлы прямо в командной строке попробуйте nano в Mac и edit в PC:

-------------------------------------------------------------------------------------

Стр. 23

\$ nano onTheMac.txt
\$ edit onThePC.txt
Эти команды запустят редактор прямо в пределах окна командной строки.

(exit)

Когда вы закончили работу с командной строкой вы можете выйти из нее простой командой
exit:
\$ exit
Если вы в Windows, то это закроет окно Command Prompt. Если вы в Mac, то команда exit 
не закроет окно, но остановит все запущенные процессы. Если вы хотите заурыть окно 
ерминала во время выполнения команды exit, перейдите к свойствам терминала (Terminal
Preferences) под настройками (Settings), затем Shell, выберите "закрыть, если Shell
пустой" ("Close if the shell exited cleanly") под "Когда выход из Shell" ("When the
shell exits").

[ PATH ]

В заключение немного теории: что именно происходит, когда вы вводите имя команды в 
командную строку? Что же, все эти команды являются простыми маленькими однозадачными
программами. Они хранятся в одной или (обычно) нескольких папках на вашем компьютере.
Как командная строка узнаёт какую папку необходимо просмотреть чтобы найти команду, 
которцю вы хотите выполнить? Для этого используется переменная \$PATH (просто PATH в
Windows). Это просто список путей до папок, которые содержат эти команды. Когда нужно
найти программу или скрипт, просто производится поиск по папкам в переменной \$PATH.
Так что же содержится в вашей переменной \$PATH? Вот содержимое моих переменных 
\$PATH (я разбил на строки для читаемости).
 
 (Mac)

 \$ echo \$PATH
/usr/local/bin:/usr/local/sbin:
/usr/local/mysql/bin:~/bin:/opt/local/sbin:
/usr/bin:/bin:/usr/sbin:/sbin:/usr/local/bin:
/usr/local/git/bin:/usr/X11/bin:
/opt/local/bin

(Windows)

\$ PATH
PATH=C:\Perl\site\bin;C:\Perl\bin;C:\Windows\system32;
C:\Windows;
C:\Windows\System32\Wbem;
C:\Program Files\Java\jdk1.6.0_18\bin;
C:\tools\;C:\Python26\;C:\Ruby186\bin

Обратите внимание, что в Windows пути разделяются точкой с запятой, в то время как
в Mac -- двоеточием. Хотя большиство из того, что вы здесь видите задано по умолчанию,
я добавил несколько путей в Mac и в Windows. Это более обширные навыки, чем мы получим 
здесь, но имейте ввиду, что такое возможно.

(Обширные навыки работы с командной строкой)

Если вы попробовали все команды, которые мы рассмотрели, то вы на пути освоения работы
за командной строкой. Если вы хотите узнать больше, есть много ресурсов. Иногда лучшим
справочником служит встроенная документация по командам. В Mac вы можете получить
доступ к документации с помощью одной из следующих команд:
\$ man <command-name>
\$ <command-name> -h
\$ <command-name> --help
\$ <command-name> help
В Windows будет работать так:
\$ <command-name> /?
Если вы в Mac или Linux и хотите больше узнаить о командной строке, я от всей души
рекомендую скринкасты PeepCode: Meeting the Command Line и Advanced Command Line.
В Windows используйте Windows Power Shell, более мощная версия командной строки от
Microsoft.

[ Итог ]

В этой главе мы изучили базовые приёмы работы с командной строкой. Сейчас мы готовы
к погружениею в Git!